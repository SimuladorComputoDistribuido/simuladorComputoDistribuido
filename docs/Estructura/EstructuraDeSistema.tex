\documentclass[a4paper,12pt]{article}

\usepackage[spanish]{babel}
\usepackage[utf8]{inputenc}
\usepackage{fancyhdr}
\usepackage{dirtree}
\usepackage{hyperref}
\usepackage[nottoc]{tocbibind}

\usepackage{graphicx}
\usepackage{amsmath}
\usepackage{amssymb}

\graphicspath{ {imagenes/} }

\usepackage{hyperref}
\hypersetup{
    colorlinks=true,
    linkcolor=black,
    filecolor=magenta,
    urlcolor=cyan,
    citecolor=black
}

\title{\bf Estructura para Proyectos de Software}
\author{Jerónimo Almeida Rodríguez}
\date{\today}

\pagestyle{fancy}
\lhead{Jerónimo Almeida}
\chead{Simulador C.D.}
\rhead{Alan Arteaga}
\lfoot{jalrod@ciencias.unam.mx}
\rfoot{alanarteagav@ciencias.unam.mx}

\begin{document}
\begin{titlepage}
    \centering
    {\scshape\Huge Universidad Nacional Autónoma de México \par}
    \vspace{2cm}
    {\scshape\huge Proyecto de Simulador de Cómputo Distribuido \par}
    \vspace{2cm}
    {\Large\textsc Jerónimo Almeida Rodríguez \par}
    \vspace{.5cm}
    {\large\texttt{ jalrod@ciencias.unam.mx}\par}
    \vspace{0.5cm}
    {\Large\textsc Alan Ernesto Arteaga Vázquez \par}
    \vspace{.5cm}
    {\large\texttt{ alanarteagav@ciencias.unam.mx}\par}
    \vspace{2cm}
    \vfill
    \begin{figure}[hb!]
        \includegraphics[width=.3\textwidth]
            {../../logos/escudo_f-ciencias.png}\hfill
        \includegraphics[width=.3\textwidth]
            {../../logos/Escudo_UNAM.png}\hfill
    \end{figure}
\end{titlepage}
%
\tableofcontents
\newpage

\section{Reglas de Trabajo.}
\subsection{Reuniones.}{
  \begin{itemize}
    \item{¿Qué se hizo desde la última reunión?}
    \item{¿Qué dificultades se encontraron?}
    \item{Definir objetivos de la reunión.}
    \item{¿Qué se va a hacer para la próxima reunión?}
  \end{itemize}
}
\subsection{Toma de decisiones.}{
  \begin{itemize}
    \item{Plantear el problema.}
    \item{Establecer el plan de acción.}
    \item{Repartir trabajo.}
  \end{itemize}
}

\section{Administración.}
\subsection{Planeación general.}{
\subsubsection{Objetivos.}{
  \begin{itemize}
    \item{Desarrollar un sistema que simule la ejecución de algoritmos
      distribuidos sobre algún conjunto de procesos.}
    \item{Crear una interfaz gráfica que represente esta simulación.}
  \end{itemize}
}
\subsubsection{Restricciones.}{
  \begin{itemize}
    \item{El sistema debe ser compatible con Mac y Linux.}
    \item{El lenguaje de programación a usar es Python 3.}
  \end{itemize}
}
\subsubsection{Fechas.}{
  \begin{center}
  \begin{tabular}{ p{0.85\textwidth} | p{0.15\textwidth}}
    Descripción & Fecha \\\hline
    Desarrollar la simulación del paso de mensajes entre $n$ procesos y el
    retorno de algún valor de aceptación. & 27/07/2020\\
  \end{tabular}
  \end{center}
}
}
\subsection{Detalles particulares.}{
\begin{itemize}
  \item{Definir actividades, quién las hace y fecha de entrega.}
  \item{Dar seguimiento al tablero.}
  \item{Actualizar Repositorio.}
  \item{Registrar Cambios.}
\end{itemize}
\subsubsection{Actividades.}{
  \begin{center}
  \begin{tabular}{  p{0.65\textwidth} | p{0.20\textwidth} | p{0.15\textwidth}}
    Descripción & Autor & Fecha \\\hline\hline
    Estructurar repositorio según se acordó (\nameref{enlaces}). & Alan &
      15/07/2020 \\\hline
    Crear carpeta de google Drive (\nameref{enlaces}). & Alan &
    15/07/2020 \\\hline
    Hacer diagrama de clases (\nameref{clases}). & Alan &
      15/07/2020 \\\hline
    Establecer estructura de archivos (\nameref{paquetes}). & Jerónimo &
      15/07/2020 \\\hline
    Dar descripción del diagrama de clases (\nameref{clases}). & Jerónimo &
      15/07/2020 \\\hline
  \end{tabular}
  \end{center}
}
\subsubsection{Cambios.}{
  \begin{center}
  \begin{tabular}{  p{0.65\textwidth} | p{0.20\textwidth} | p{0.15\textwidth}}
    Descripción & Autor & Fecha \\\hline\hline
  \end{tabular}
  \end{center}
}
}

\section{Requerimientos.}
\begin{itemize}
  \item{Modelar un proceso que envíe y reciba mensajes. Una clase
    \texttt{proceso} simula al proceso y contiene la funcinalidad para recibir y
    enviar mensajes.
  }
  \item{Una clase que según los parámetros recibidos, construya una simulación
    del sistema utilizando los procesos anteriormente mencionados.
  }
  \item{\textbf{Pendiente} Paso de mensajes y sincronización.}
\end{itemize}

\section{Diseño.}
\begin{itemize}
  \item{Identificar componentes.}
  \item{Definir interacción e integración de las componentes.
    \begin{enumerate}
      \item{Abstraer.}
      \item{Separar conceptos.}
      \item{Modularizar.}
      \item{Acoplar.}
    \end{enumerate}
  }
  \item{Definir arquitectura.}
  \item{Identificar lenguaje(s) y herramientas.}
  \item{Hacer diagrama de paquetes.}
  \item{Identificar clases y sus relaciones.}
  \item{Hacer diagramas de secuencia y de navegación.}
\end{itemize}
\subsection{Componentes.}{
\begin{itemize}
  \item{Proceso.}
  \item{Constructor del sistema. Permite instanciar procesos, pasar parámetros e
    iniciar la simulación.
  }
\end{itemize}
}
\subsection{Arquitectura.}{
  MVC.\\
  Primera entrega consiste solamente de una parte del modelo (procesos y
  constructor).
}
\subsection{Lenguaje(s) y herramientas.}{
\subsubsection{Lenguaje(s).}{
  \texttt{Python 3: }\url{https://www.python.org/}\\
}
\subsubsection{Herramientas.}{
  \texttt{Simpy: } \url{https://simpy.readthedocs.io/en/latest/contents.html}
    \cite{Simpy}\\
  \texttt{NetworkX: } \url{https://networkx.github.io/}\\
  \texttt{NetworkX Viewer: } \url{https://pypi.org/project/networkx_viewer/}\\
}
}
\subsection{Diagrama de paquetes.}{
  \label{paquetes}
  \dirtree{%
    .1 simuladorComputoDistribuido.
      .2 README$.$md  .
      .2 simulador.
        .3 LICENSE.
        .3 README$.$md .
        .3 simulador.
          .4 proceso$.$py.
          .4 constructor$\_$del$\_$sistema$.$py.
        .3 setup.
        .3 tests.
          .4 prueba$\_$unitaria$.$py.
      .2 documentos.
        .3 EstructuraDeSistema$.$tex .
  }
}
\subsection{Diagrama de clases.}{
  \label{clases}
}

\section{Desarrollo.}
\subsection{Preparación.}{
  \begin{itemize}
    \item{Instalar herramientas de desarrollo.}
    \item{Definir estrategia de construcción.}
    \item{Establecer estándar de documentación.}
  \end{itemize}
}
\subsection{Programación.}{
  \begin{itemize}
    \item{Generar el código de cada componente siguiendo estándares y los
    principios de modularidad y anticipando posibles cambios.
    }
    \item{Coordinar cambios con el resto de los integrantes.}
    \item{Desarrollar pruebas y probar código.}
  \end{itemize}
}

\section{Extras.}
\subsection{Enlaces.}{
  \label{enlaces}
  \begin{itemize}
    \item{\textbf{Repositorio: }  \url{https://github.com/SimuladorComputoDistribuido/simuladorComputoDistribuido/}}
    \item{\textbf{Google Drive: } \url{https://drive.google.com/drive/folders/1TPISkVK6i3M-eNBr38q2rvlAkLRuWCDY}}
  \end{itemize}
}
\subsection{Convenciones.}{
\subsubsection{Documentación.}{
  La documentación del código es generada con el paquete pdoc
  (\url{https://pdoc3.github.io/pdoc/}) y está generada según el estándar
  establecido en la norma PEP-257
  (\url{https://www.python.org/dev/peps/pep-0257/}).\\
  Los archivos README siguen el estándar de GitHub Flavored Markdown
  (\url{https://github.github.com/gfm/}).
}

\subsubsection{Formatos.}{
  \begin{itemize}
    \item{\textbf{Formato de fecha:} dd/mm/aaaa}
    \item{\textbf{Campo de autor:} Primer nombre seguido de la primera letra de
      los apellidos en caso de que los participantes tengan el mismo nombre.
    }
  \end{itemize}
}
\subsection{Notas}{
  \begin{enumerate}
    \item{Esta plantilla esta basada en las notas de clase del curso de
      ingeniería del software impartido por Guadalupe Ibargüengoitia G. y Hanna
      Oktaba.\\
      Abarca el material cubierto en las presentaciones 3 (Prácticas sociales y
      de trabajo en equipo) - 7 (Construcción de software).
    }
  \end{enumerate}
}
}

\bibliographystyle{acm}
\bibliography{bibliografia}

\end{document}
